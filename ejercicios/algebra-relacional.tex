\documentclass{db-practice}

\begin{document}

\title{Ejercicios de Álgebra Relacional}

\section{Editoriales}

Sean las relaciones siguientes:

\begin{addmargin}[1.5em]{0em}
    \texttt{EDITORIALES(\underline{E\#}, Nombre, Ciudad)}\\
    \texttt{LIBROS(\underline{L\#}, Título, Autor, Año)}\\
    \texttt{PAPELERÍAS(\underline{P\#}, Nombre, Ciudad)}\\
    \texttt{ELP(\underline{E\#}, \underline{L\#}, \underline{P\#}, Cantidad)}
\end{addmargin}


Se pide escribir en álgebra relacional las respuestas a las preguntas siguientes:

\begin{enumerate}
    \item Obtener los nombres de las papelerías abastecidas por alguna editorial de `Madrid'.
    \item Obtener los valores de \texttt{E\#} para las editoriales que suministran a las papelerías P1 y P3 libros publicados en el año 1978.
    \item Obtener los valores de \texttt{P\#} de las papelerías abastecidas completamente por la editorial E1.
    \item Obtener los valores de \texttt{L\#} para los libros suministrados por todas las papelerías que no sean de `Madrid'.
\end{enumerate}

\section{Programas}

Dada la base de datos compuesta por las siguientes relaciones:

\begin{addmargin}[1.5em]{0em}
    \texttt{PROGRAMAS(\underline{P\#}, Memoria, SO, Distribuidor)}\\
    \texttt{USUARIOS(\underline{U\#}, Edad, Sexo)}\\
    \texttt{ORDENADORES(\underline{O\#}, Modelo, SO, Capacidad)}\\
    \texttt{USOS(\underline{U\#}, \underline{P\#}, \underline{O\#}, Tiempo)}
\end{addmargin}

Se pide expresar en términos de álgebra relacional la secuencia de operaciones necesaria para efectuar las siguientes consultas a la base de datos:

\begin{enumerate}
    \item Obtener los usuarios (\texttt{U\#}) que usan al menos todos los programas del distribuidor `D1'.
    \item Obtener los programas (\texttt{P\#}) que sólo son usados por el usuario `U5'.
    \item Obtener distribuidores que venden los programas `P5' y `P8'.
    \item Obtener los modelos de los ordenadores que son usados por personas mayores de 30 años durante más de 3 horas.
\end{enumerate}

\section{Videoteca}

Sean las relaciones siguientes:

\begin{addmargin}[1.5em]{0em}
    \texttt{SOCIO(\underline{Aficionado}, \underline{Videoclub})}\\
    \texttt{GUSTA(\underline{Aficionado}, \underline{Película})}\\
    \texttt{VIDEOTECA(\underline{Videoclub}, \underline{Película})}
\end{addmargin}

Se pide escribir en álgebra relacional las sentencias necesarias para responder a las preguntas siguientes:

\begin{enumerate}
    \item Películas que le gustan al aficionado `José Pérez'.
    \item Videoclubes que disponen de alguna película que le guste al aficionado `José Pérez'.
    \item Aficionados que son socios de al menos de un videoclub que dispone de alguna película de su gusto.
    \item Aficionados que no son socios de ningún videoclub donde tengan alguna película de su gusto.
\end{enumerate}

\section{Maquinaria}

Dada la base de datos formada por las siguientes tablas:

\begin{addmargin}[1.5em]{0em}
    \texttt{MÁQUINAS(\underline{M\#}, Tipo, Matrícula, PrecioHora)}\\
    \texttt{FINCAS(\underline{F\#}, Nombre, Extensión)}\\
    \texttt{TRABAJADOR(\underline{T\#}, Nombre, Dirección)}\\
    \texttt{PARTES(\underline{T\#}, \underline{M\#}, \underline{F\#}, \underline{Fecha}, TipoFaena, Tiempo)}
\end{addmargin}

Se pide dar soluciones algebraicas a las siguientes consultas:

\begin{enumerate}
    \item Obtener todos los \texttt{T\#} que usan todas las máquinas de tipo 1.
    \item Obtener todos los \texttt{F\#} para aquellas fincas en las que han realizado trabajos las máquinas `M1' y `M3'.
    \item Obtener el valor de \texttt{M\#} para aquellas máquinas que no han sido utilizadas nunca en ningún trabajo.
    \item Obtener todos los nombres de fincas en las que se ha trabajado más de 5 horas con máquinas cuyo precio por hora sea superior a 25€.
\end{enumerate}

\section{Prácticas}

Dada la base de datos compuesta por las siguientes tablas:

\begin{addmargin}[1.5em]{0em}
    \texttt{ALUMNOS(\underline{A\#}, Nombre, Grupo)}\\
    \texttt{PRÁCTICAS(\underline{P\#}, Curso, Fecha)}\\
    \texttt{ENTREGA(\underline{A\#}, \underline{P\#}, Nota)}
\end{addmargin}

Se pide dar solución en álgebra relacional a las consultas:

\begin{enumerate}
    \item Obtener los nombres de los alumnos que han aprobado todas las prácticas de tercer curso.
    \item Obtener los nombres de los alumnos que han entregado todas las prácticas de tercer curso.
    \item Obtener los alumnos que han entregado prácticas de segundo y tercer curso.
    \item Obtener los alumnos que sólo han entregado prácticas de segundo curso.
    \item Obtener los alumnos que han entregado prácticas de segundo curso y pertenecen al grupo `BD-11'.
    \item Obtener el nombre de los alumnos que no han suspendido ninguna práctica de las que han entregado.
\end{enumerate}

\section{Ciclismo}

La Federación Internacional de Ciclismo Profesional desea tener una Base de Datos Relacional (BDR) con las siguientes tablas:

\begin{addmargin}[1.5em]{0em}
    \texttt{EQUIPOS(\underline{E\#}, Nombre, País)}\\
    \texttt{CICLISTAS(\underline{C\#}, Nombre, E\#)}\\
    \texttt{COMPETICIONES(\underline{M\#}, Nombre, País, Duración)}\\
    \texttt{CLASIFICACIÓN(\underline{M\#}, \underline{C\#}, Puesto)}
\end{addmargin}

Se pide escribir las sentencias necesarias en álgebra relacional para:

\begin{enumerate}
    \item Obtener los ciclistas que sólo han participado en competiciones de duración inferior a 15 días.
    \item Obtener los ciclistas de equipos españoles que han competido en todas las competiciones de España.
    \item Obtener los ciclistas que han obtenido un primer y un segundo puestos en competiciones con una duración inferior a 15 días.
\end{enumerate}

\section{Infracciones de tráfico}

Dada las tablas siguientes:

\begin{addmargin}[1.5em]{0em}
    \texttt{CONDUCTOR(\underline{C\#}, DNI, Nombre)}\\
    \texttt{AGENTE(\underline{A\#}, Nombre, Rango)}\\
    \texttt{INFRACCIÓN(\underline{I\#}, Descripción, Importe)}\\
    \texttt{DENUNCIA(\underline{C\#}, \underline{A\#}, \underline{I\#}, \underline{Fecha}, Pagada)}
\end{addmargin}

Se pide escribir en álgebra relacional las sentencias necesarias para:

\begin{enumerate}
    \item Obtener el nombre de aquellos conductores que hayan sido denunciados por todas las infracciones inferiores a 600€.
    \item Obtener el código de aquellos agentes que sólo hayan denunciado infracciones de `Estacionamiento' (atributo \texttt{Descripción}).
    \item Obtener el código de aquellos conductores que no tengan ninguna denuncia pendiente de pago (atributo \texttt{Pagada}).
\end{enumerate}

\end{document}
