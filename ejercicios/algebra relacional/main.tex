\documentclass[a4paper]{article}

\usepackage[utf8]{inputenc}
\usepackage[a4paper, margin=3.5cm]{geometry}


\usepackage{graphicx}
\usepackage{url}
\usepackage[spanish]{babel}
\usepackage{fancyhdr}

\renewcommand{\headrulewidth}{0.6pt}
\renewcommand{\footrulewidth}{0.6pt}

\pagestyle{fancy}
\setlength\headheight{50pt}
\lhead{\includegraphics[height=1.5cm]{figs/upm_logo}}
\chead{Bases de Datos\\\vspace{.5em} Problemas de álgebra relacional\\\vspace{-.1em}}
\rhead{\includegraphics[height=1.5cm]{figs/etsisi_logo}}
\lfoot{\textbf{Tema 3:} Modelo relacional}
\cfoot{}
\rfoot{\thepage}

\parskip 1.1ex % paragraph spacing
\setlength{\parindent}{0pt} % no indent

\begin{document}

\section{Editoriales}

Sean las relaciones siguientes:

\texttt{EDITORIALES(\underline{E\#}, Nombre, Ciudad)}

\texttt{LIBROS(\underline{L\#}, Título, Autor, Año)}

\texttt{PAPELERÍAS(\underline{P\#}, Nombre, Ciudad)}

\texttt{ELP(\underline{E\#}, \underline{L\#}, \underline{P\#}, Cantidad)}

Se pide escribir en álgebra relacional las respuestas a las preguntas siguientes:

\begin{itemize}
    \item Obtener los nombres de las papelerías abastecidas por alguna editorial de "Madrid".
    \item Obtener los valores de E\# para las editoriales que suministran a las papelerías P1 y P3 libros publicados en el años 1978.
    \item Obtener los valores de P\# de las papelerías abastecidas completamente por la editorial E1.
    \item Obtener los valores de L\# para los libros suministrados por todas las papelerías que no sean de "Madrid".

\end{itemize}

\end{document}
